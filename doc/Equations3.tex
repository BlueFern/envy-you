\section{Equations}

\subsection{Astrocyte Model}

\subsubsection{Scaling}
%
The  \gls{AC}  volume-area ratio (in $m$):
\begin{equation} \label{eq:R_k}
\dfrac{\mathrm{d}R_k}{\mathrm{d}t}= L_p \left( Na_k+K_k+Cl_k+HCO_{3_k}-Na_s-K_s-Cl_s-HCO_{3_{s}}+\frac{X_k}{R_k}\right)
\end{equation}
%
The \gls{SC} volume-surface ratio  (in $m$):
\begin{equation} \label{eq:R_tot}
R_s = R_{tot} - R_k  
\end{equation}
%
\begin{table}[h!]
\centering
\begin{tabular}{| p{0.09\linewidth} | >{\footnotesize} p{0.6\linewidth} | >{\footnotesize} p{0.17\linewidth} | >{\footnotesize} p{0.02\linewidth} |}
\arrayrulecolor{lightgrey}\hline
$L_p$ 			& The total water permeability per unit area of the astrocyte 			& 2.1e-9 \mperuMs &  \cite{Ostby2009}  \\
$X_k$			& Number of negatively charged impermeable ions trapped within the astrocyte divided by the astrocyte membrane area								& 12.41e-3 \uMm & \cite{Ostby2009}  \\
$R_{tot}$ 		& Total volume surface ratio AC+SC   		& 8.79e-8 \m & \cite{Ostby2009}  \\
\hline
\end{tabular}
\end{table}
\newpage
\subsubsection{Input Signal} \label{sec:InputSignal}
%
The neuronal \gls{K} input signal (-):\\
%
For $ t<t_0$ and $t>t_3$:
\begin{equation}
f(t)=0
\end{equation}
%
For $ t_0 \leq t \leq t_1$:
\begin{equation}
f(t)=F_{input} \dfrac{(\alpha+\beta-1)!}{(\alpha-1)!(\beta-1)!} \left( \dfrac{1-(t-t_0)}{\Delta t}\right) ^{\beta -1} \left( \dfrac{t-t_0}{\Delta t}\right) ^{\alpha -1} 
\end{equation}
%
For $ t_1 \leq t \leq t_2$:
\begin{equation}
f(t)=0
\end{equation}
%
For $ t_2 \leq t \leq t_3$:
\begin{equation}
f(t)=-F_{input}
\end{equation}
%
\paragraph{$\rho$ input signal}
The smooth pulse function $\rho$
\begin{equation}
\rho(t) = \frac{Amp - base}{2}\times\left(1+\mathrm{tanh}\left(\frac{t-t_0}{\theta_L}\right)\right)+base+\frac{Amp-base}{2}\times\left(1+\mathrm{tanh}\left(\frac{t-t_2}{\theta_R}\right)\right)+base-Amp                      
\end{equation}
%
\begin{table}[h!]
\centering
\begin{tabular}{| p{0.09\linewidth} | >{\footnotesize} p{0.6\linewidth} | >{\footnotesize} p{0.17\linewidth} | >{\footnotesize} p{0.02\linewidth} |}
\arrayrulecolor{lightgrey}\hline
$t_0$ 			& Start of neuronal pulse	& 200 s &  ME  \\
$t_1$ 			& End of neuronal pulse	& 210 s &  ME  \\
$t_2$ 			& Start of back-buffering	& 230 s &  ME  \\
$t_3$ 			& End of back-buffering	& 240 s &  ME  \\
$F_{input}$ 	& Amplitude scaling factor 	& 2.5 	&  ME  \\
$\alpha$ 		& Beta-distribution constant	& 2 	&  ME  \\
$\beta$ 		& Beta-distribution constant	& 5 	&  ME  \\
$\Delta t$ 		& Time-scaling factor	& 10 s	&  ME  \\
$Amp$           & Amplitude of smooth pulse function & 0.7 & ME\\
$base$          & Baseline of smooth pulse function & 0.1 & ME\\
$\theta_L$      & Left ramp of smooth pulse function & 1 & ME\\
$\theta_R$      & Left ramp of smooth pulse function & 1 & ME\\
   
\hline
\end{tabular}
\end{table}

% 
\subsubsection{Conservation Equations}
\paragraph{Synaptic Cleft}~\\
%
\gls{K} concentration in the \gls{SC} (times the \gls{SC} volume-area ratio $R_s$; in \uMm):
\begin{equation} \label{eq:KEx}
\dfrac{\mathrm{d}N_{K_s}}{\mathrm{d}t}= k_C f(t) -\dfrac{\mathrm{d}N_{K_k}}{\mathrm{d}t} - J_{BK_k} 
\end{equation}
%
\gls{Na} concentration in the \gls{SC}  (times the \gls{SC} volume-area ratio $R_s$; in \uMm):
\begin{equation} \label{eq:NaEx}
\dfrac{\mathrm{d}N_{Na_s}}{\mathrm{d}t}= - k_C f(t) -\dfrac{\mathrm{d}N_{Na_k}}{\mathrm{d}t}
\end{equation}
%
\gls{HCO3} concentration in the SC  (times the \gls{SC} volume-area ratio $R_s$; in \uMm):
\begin{equation} \label{eq:HCOEx}
\dfrac{\mathrm{d}N_{HCO_{3_{s}}}}{\mathrm{d}t}=-\dfrac{\mathrm{d}N_{HCO_{3_{k}}}}{\mathrm{d}t}
\end{equation}
\begin{table}[h!]
\centering
\begin{tabular}{| p{0.09\linewidth} | >{\footnotesize} p{0.6\linewidth} | >{\footnotesize} p{0.17\linewidth} | >{\footnotesize} p{0.02\linewidth} |}
\arrayrulecolor{lightgrey}\hline
$ k_C $  & Input scaling parameter & 7.35e-5 \muMps & \cite{Ostby2009} \\
\hline
\end{tabular}
\end{table}

\paragraph{Astrocyte}~\\
%
\gls{K} concentration in the AC  (times the AC volume-area ratio $R_k$; in \uMm):
\begin{equation} \label{eq:KInt}
\dfrac{\mathrm{d}N_{K_k}}{\mathrm{d}t}=- J_{K_k} + 2 J_{NaK_{k}} + J_{NKCC1_{k}} +  J_{KCC1_{k}}
- J_{BK_k}  
\end{equation}
%
\gls{Na} concentration in the AC  (times the AC volume-area ratio $R_k$; in \uMm):
\begin{equation} \label{eq:NaInt}
\dfrac{\mathrm{d}N_{Na_k}}{\mathrm{d}t}=-J_{Na_k} - 3 J_{NaK_{k}} + J_{NKCC1_{k}} +  J_{NBC_{k}}
\end{equation}
%
\gls{HCO3} concentration in the AC  (times the AC volume-area ratio $R_k$; in \uMm):
\begin{equation} \label{eq:HCOInt}
\dfrac{\mathrm{d}N_{HCO_{3_k}}}{\mathrm{d}t}= 2 J_{NBC_{k}} 
\end{equation}
%
\gls{Cl} concentration in the AC  (times the AC volume-area ratio $R_k$; in \uMm):
\begin{equation} \label{eq:ClInt}
\dfrac{\mathrm{d}N_{Cl_k}}{\mathrm{d}t}= \dfrac{\mathrm{d}N_{Na_k}}{\mathrm{d}t} + \dfrac{\mathrm{d}N_{K_k}}{\mathrm{d}t} - \dfrac{\mathrm{d}N_{HCO_{3_{k}}}}{\mathrm{d}t}
\end{equation}
%
\gls{Ca} concentration in the astrocytic cytosol:
\begin{equation} \label{eq:ckInt}
\dfrac{\mathrm{d}c_k}{\mathrm{d}t}= B_{\mathrm{cyt}}(J_{\mathrm{IP_3}}-J_{\mathrm{pump}}+J_{\mathrm{ER_leak}})
\end{equation}
%
\gls{Ca} concentration in the astrocytic \gls{ER}:
\begin{equation} \label{eq:skInt}
\dfrac{\mathrm{d}s_k}{\mathrm{d}t}= \frac{1}{VR_{\mathrm{ER_cyt}}}(\frac{dc_k}{dt})
\end{equation}
%
The inactivation variable for \gls{IP3}:
\begin{equation} \label{eq:hkInt}
\dfrac{\mathrm{d}h_k}{\mathrm{d}t}= k_{\mathrm{on}}[K_{\mathrm{inh}}-(c_k+K_{\mathrm{inh}})h_k]
\end{equation}
%
The \gls{IP3} concentration:
\begin{equation} \label{eq:ikInt}
\dfrac{\mathrm{d}i_k}{\mathrm{d}t}= r_hG-k_{\mathrm{deg}}i_k
\end{equation}
%
The EET concentration:
\begin{equation} \label{eq:eetkInt}
\dfrac{\mathrm{d}eetk_k}{\mathrm{d}t}= V_{\mathrm{eet}}(c_k-c_{\mathrm{k,min}})-k_{\mathrm{eet}}eet_k
\end{equation}
%
Open probability of the BK channel (\pers):
\begin{equation} \label{eq:dwkdt}
\frac{\mathrm{d}w_{k}}{\mathrm{d}t} = \phi_{w} \left(w_{\infty}-w_{k} \right) 
\end{equation}
%
\begin{table}[h!]
	\centering
	\begin{tabular}{| p{0.09\linewidth} | >{\footnotesize} p{0.6\linewidth} | >{\footnotesize} p{0.17\linewidth} | >{\footnotesize} p{0.02\linewidth} |}
		\arrayrulecolor{lightgrey}\hline
		$ VR_{\mathrm{ER_{\mathrm{cyt}}}} $  & Volume ratio of the \gls{ER} to the cytosol in the astrocyte  & 0.185 [-] & \cite{} \\
		$k_{\mathrm{on}} $         & Rate of \gls{Ca} binding to the \gls{IP3}R  & 2 [\uMps] &  \cite{} \\
		$K_{\mathrm{inh}}$          & Dissociation rate of $k_{\mathrm{on}}$  & 0.1[\uM] & \cite{} \\
		$r_h$                      & Maximum rate of \gls{IP3} production in the astrocyte  &  4.8 [\uM] & \cite{} \\
		$k_{\mathrm{deg}}$          & Rate constant for \gls{IP3} degradation  &  1.25 [\pers] & \cite{} \\
		$V_{\mathrm{eet}} $        & Rate constant for EET production   & 72 [\uM] & \cite{} \\
		$k_{\mathrm{eet}}$         & Rate constant for EET degradation  & 7.2 [\uM] & \cite{} \\
		$c_{\mathrm{k,min}}$       & Minimum \gls{Ca} concentration required for EET production & 0.1 [uM] & \cite{} \\			
		\hline
	\end{tabular}
\end{table}
%
\paragraph{Perivascular Space}~\\
\gls{K} concentration in the PVS  (in \uM):
\begin{equation} \label{eq:K_p}
\dfrac{\mathrm{d}K_{p}}{\mathrm{d}t}= \frac{J_{BK_k}}{R_k VR_{pa}} + \frac{J_{KIR_i}}{ VR_{ps}}
\end{equation}
%
\begin{table}[h!]
\centering
\begin{tabular}{| p{0.09\linewidth} | >{\footnotesize} p{0.6\linewidth} | >{\footnotesize} p{0.17\linewidth} | >{\footnotesize} p{0.02\linewidth} |}
\arrayrulecolor{lightgrey}\hline
$ VR_{pa} $  & Volume ratio of PVS to AC & 0.001 [-] & \cite{LoesEvert} \\
$ VR_{ps} $  & Volume ratio of PVS to SMC & 0.001 [-] & \cite{LoesEvert} \\
\hline
\end{tabular}
\end{table}

\subsubsection{Fluxes}~\\
%
\gls{K} flux (times the AC volume-area ratio $R_k$; in \uMmps): 
\begin{equation} \label{eq:J_K}
J_{K_k}=\frac{g_{K_{k}}}{F}(v_k - E_{K_k}) C_{correction}
\end{equation}
%
\gls{Na} flux (times the AC volume-area ratio $R_k$; in \uMmps):
\begin{equation} \label{eq:J_Na}
J_{Na_k}=\frac{g_{Na_{k}}}{F}(v_k - E_{Na_k}) C_{correction}
\end{equation}
%
\gls{Na} and \gls{HCO3} flux through the NBC channel  (times the AC volume-area ratio $R_k$; in \uMmps): 
\begin{equation} \label{eq:J_NBC}
J_{NBC_k}=\frac{g_{NBC_k}}{F}\left(  v_k -E_{NBC_k}  \right) C_{correction}
\end{equation}
%
\gls{Cl} and \gls{K} flux through the KCC1 channel  (times the AC volume-area ratio $R_k$; in \uMmps): 
\begin{equation} \label{eq:J_KCC1}
J_{KCC1_k}=C_{input}\frac{g_{KCC1_k}}{F}\frac{RT}{F}ln \left(\frac{K_s Cl_s }{K_k Cl_k}\right) C_{correction}
\end{equation}
%
\gls{Na}, \gls{K} and \gls{Cl} flux through the NKCC1 channel   (times the AC volume-area ratio $R_k$; in \uMmps): 
\begin{equation} \label{eq:J_NKCC1}
J_{NKCC1_k}=C_{input}\frac{g_{NKCC1_k}}{F}\frac{RT}{F}ln \left(\frac{Na_s K_s {Cl_s}^2}{Na_k K_k {Cl_k}^2}\right) C_{correction}
\end{equation}
%
Flux through the sodium potassium pump   (times the \gls{AC} volume-area ratio $R_k$; in \uMmps): 
\begin{equation} \label{eq:J_NaK_s}
J_{NaK_{k}}=J_{NaK_{max}}\frac{{Na_k}^{1.5}}{{Na_k}^{1.5}+{K_{Na_k}}^{1.5}}\frac{K_s}{K_s+K_{K_s}}
\end{equation}
%
\gls{K} flux through the BK channel  (times the \gls{AC} volume-area ratio $R_k$; in \uMmps): 
\begin{equation} \label{eq:J_BK}
J_{BK_k}=\frac{g_{BK_k}}{F}w_k\left( v_k-E_{BK_k} \right) C_{correction}
\end{equation}
%
\gls{Ca} flux from the ER to the cytosol in the astrocyte through \gls{IP3} Receptors (\gls{IP3}R) by \gls{IP3}: 
\begin{equation} \label{eq:J_ip3}
J_{\mathrm{IP3}}=J_{\mathrm{max}}[(\frac{i_k}{i_k+K_i})(\frac{c_k}{c_k+K_{\mathrm{act}}})h_k]^3\times [1-\frac{c_k}{s_k}] 
\end{equation}
%
The leakage \gls{Ca}  flux from the \gls{ER} to the cytosol in the astrocyte:
\begin{equation} \label{eq:J_ER_leak}
J_{\mathrm{ER_leak}} = P_L(1-\frac{c_k}{s_k})
\end{equation}	
%
The ATP dependent \gls{Ca}  pump flux from the cytoplasm to the ER in the astrocyte:
\begin{equation} \label{eq:J_pump}
J_{\mathrm{pump}} = V_{\mathrm{max}}\frac{c_k^2}{c_k^2+k_pump^2}
\end{equation}
%
%
%
\begin{table}[h!]
\centering
\begin{tabular}{| p{0.09\linewidth} | >{\footnotesize} p{0.6\linewidth} | >{\footnotesize} p{0.17\linewidth} | >{\footnotesize} p{0.02\linewidth} |}
\arrayrulecolor{lightgrey}\hline	
$F$ 			& Faradays constant														& 9.649e4 \Cmol 	& \\
$R_g$ 			& Gas constant 															& 8.315 \JmolK		& \\
$T$ 	    	& Temperature 															& 300 \Kelvin		& \\
$g_{K_{k}}$ 	& Specific ion conductance of potassium 								& 40 \perOhmm 		& \cite{Ostby2009}  \\
$g_{Na_k}$ 		& Specific ion conductance of sodium 									& 1.314  \perOhmm 	& \cite{Ostby2009}  \\
$g_{NBC_k}$ 	& Specific ion conductance of the NBC cotransporter						& 7.57e-1 \perOhmm 	& \cite{Ostby2009}  \\
$g_{KCC1_k}$ 	& Specific ion conductance of the KCC1 cotransporter					& 1e-2 \perOhmm 	& \cite{Ostby2009}  \\
$g_{NKCC1_k}$ 	& Specific ion conductance of the NKCC1 cotransporter	 				& 5.54e-2 \perOhmm 	& \cite{Ostby2009}  \\
$J_{NaK_{max}}$ & The maximum flux through the NaKATPase pump							& 1.42e-3 \uMms 	& \cite{Ostby2009}  \\
$g_{BK_k}$ 		& Specific ion conductance of the BK channel							& 1.16     \perOhmm & \cite{LoesEvert}  \\
$C_{correction}$  & correction factor                                                   & 10e3 [-]        & \cite{LoesEvert} \\
$C_{input}$     & Block function to switch the channel on and off                       & 0 ; 1 [-]         & \cite{LoesEvert} \\
$J_{max}$       & Maximum \gls{IP3} rate                                                & 2880 \uMps        & \cite{Farr2011} \\  
$K_I$           & Dissociation constant for \gls{IP3} binding to \gls{IP3}R             & 0.03 \uM          & \cite{Farr2011} \\ 
$K_{act}$       & Dissociation constant for \gls{Ca} binding to \gls{IP3}R              & 0.17 \uM          & \cite{Farr2011} \\ 
$P_L$           & Associated with the steady state \gls{Ca} balance                     & 5.2 \uM           & \cite{Farr2011} \\ 
$V_{max}$       & Maximal pumping rate of the \gls{Ca} pump                             & 20 \uMps          & \cite{Farr2011} \\ 
$k_{pump}$      & Dissociation constant of the \gls{Ca} pump                            & 0.24 \uM          & \cite{Farr2011} \\ 
\hline
\end{tabular}
\end{table}



\newpage
\subsubsection{Additional Equations}
\paragraph{Synaptic Cleft}~\\
%
\gls{Cl} concentration  (times the SC volume-area ratio $R_s$; in \uMm): 
\begin{equation} \label{eq:ClEx}
N_{Cl_s}= N_{Na_s}+N_{K_s}-N_{ HCO_{3_s}}
\end{equation}

\paragraph{Astrocyte}~\\
%
Membrane voltage of the \gls{AC} (mV):
\begin{equation} \label{eq:v_k}
v_k=\frac{g_{Na_k}E_{Na_k}+g_{K_k}E_{K_k}+g_{Cl_k}E_{Cl_k}+g_{NBC_k}E_{NBC_k} + g_{BK_k}w_kE_{BK_k} -J_{NaK_k}F C_{correction} }{ g_{Na_k}+g_{K_k}+g_{Cl_k}+g_{NBC_k}+g_{BK_k}w_k }
\end{equation}
%
Nernst potential for the potassium channel (in mV):
\begin{equation} \label{eq:E_K}
E_{K_k}=\frac{R_gT}{z_K F}ln\left( \frac{K_s}{K_k}\right) 
\end{equation}
%
Nernst potential for the sodium channel (in mV):
\begin{equation} \label{eq:E_Na}
E_{Na_k}=\frac{R_gT}{z_{Na} F}ln\left( \frac{Na_s}{Na_k}\right) 
\end{equation}
%
Nernst potential for the chloride channel (in mV):
\begin{equation} \label{eq:E_Cl}
E_{Cl_k}=\frac{R_gT}{z_{Cl} F}ln\left( \frac{Cl_s}{Cl_k}\right) 
\end{equation}
%
Nernst potential for the NBC channel (in mV):
\begin{equation} \label{eq:E_NBC}
E_{NBC_k}=\frac{R_gT}{z_{NBC} F}ln\left( \frac{Na_s {HCO_{3_s}}^2}{Na_k {HCO_{3_k}}^2}\right) 
\end{equation}
Nernst potential for the BK channel (in mV):
\begin{equation} \label{eq:E_BK}
E_{BK_k}=\frac{R_gT}{z_K F}ln\left( \frac{K_p}{K_k}\right) 
\end{equation}
The Calcium buffering parameter in the astrocytic cytosol (-)
 \begin{equation} \label{eq:B_cyt}
 	B_{cyt}=\left(1+BK_{end}+ \frac{K_{ex}B_{ex}}{(K_{ex}+c_k)^2}\right)^{-1} 
 \end{equation}
The ratio of active to total G-protein (-)
\begin{equation} \label{eq:G}
   G=\frac{\rho+\delta}{K_g+\rho+\delta}
\end{equation}
Equilibrium state BK-channel (-):
\begin{equation} \label{eq:winf}
w_{\infty}=0.5 \left(1+\mathrm{tanh}\left(\frac{v_{k}+(eet_{\mathrm{shift}}eet_k)-v_{3} }{v_{4}} \right)  \right) 
\end{equation}
%
The time constant associated with the opening of BK channels	 (in \pers):
\begin{equation} \label{eq:phin}
\phi_{w}=\psi_{w}\mathrm{cosh}\left( \frac{v_{k}-v_{3}}{2v_{4}}\right) 
\end{equation}
\gls{Ca} dependent shift of the opening of the BK-channels
\begin{equation} \label{eq:v_3}
v_{3}=\frac{v_5}{2}\mathrm{tanh}\left( \frac{c_k-Ca_3}{Ca_4}\right)+v_6 
\end{equation}

\begin{table}[h!]
\centering
\begin{tabular}{| p{0.09\linewidth} | >{\footnotesize} p{0.6\linewidth} | >{\footnotesize} p{0.17\linewidth} | >{\footnotesize} p{0.02\linewidth} |}
\arrayrulecolor{lightgrey}\hline
$g_{Cl_k}$ 		& Specific ion conductance of chloride 									& 8.797e-1 [\perOhmm] & \cite{Ostby2009}  \\
$z_K$			& Valence of a potassium ion										& 1  [-] & \\ 
$z_Na$			& Valence of a sodium ion											& 1  [-] & \\ 
$z_Cl$			& Valence of a chloride ion											& -1 [-] & \\ 
$z_{NBC}$ 		& Effective valence of the NBC cotransporter complex 				& -1 [-] & \\
$BK_{end}$      & Cytosolic endogenous buffer constant                              & 40 [-] & \cite{LoesEvert} \\
$K_{ex}$        & Cytosolic exogenous buffer dissociation constant                  & 0.26 [\uM] & \cite{LoesEvert} \\
$B_{ex}$        & Concentration of cytosolic exogenous buffer                       & 11.35 [\uM] & \cite{LoesEvert} \\
$\delta$        & Ratio of the activities of the bound and unbound receptors        & 1.235e-3 [-] & \cite{Farr2011}\\
$K_G$           & The G-protein dissociation constant                               & 8.82  [uM] & \cite{Farr2011}\\
$v_{4}$			& A measure of the spread of the distribution of the open probability of the BK channel	& 14.5e-3 [\Volt]   &  \cite{Gonzalez1994}  
\\
$v_{5}$			& Determines the range of the shift of $n_{\inf}$ as calcium varies    		& 8e-3 [\Volt]  & \cite{Farr2011}  \\
$v_{6}$			& The voltage associated with the opening of half the population		& -15e-3 [\Volt]  & \cite{Farr2011}  \\
$ \psi_{w}$    	& A characteristic time for the open probability of the BK channel		& 2.664 [$s^{-1}$] & \cite{Gonzalez1994} \\

\hline
\end{tabular}
\end{table}


\newpage
\subsection{SMC and EC Model}


\subsubsection{Conservation Equations}
\paragraph{Smooth Muscle Cell}~\\
%
The cytosolic [\gls{Ca}] in the \gls{SMC} (in \uM):
\begin{equation}\label{eq:ci}
\begin{split}
\dfrac{\mathrm{d}\CaConsc}{\mathrm{d}t} = J_{IP_{3i}} - J_{SR_{uptake_{i}}} + J_{CICR_{i}} - J_{extrusion_{i}} +  J_{SR_{leak_{i}}}\dots \\
 - J_{VOCC_{i}} + J_{Na/Ca_{i}}  + 0.1J_{stretch_{i}} + J_{Ca^{2+}-coupling_{i}}^{SMC-EC}
\end{split} 
\end{equation}
%
The [Ca$^{2+}$] in the \gls{SR} of the \gls{SMC} (in \uM):
\begin{equation} \label{eq:si}
\dfrac{\mathrm{d}\CaConse}{\mathrm{d}t} =  J_{SR_{uptake_{i}}} - J_{CICR_{i}} - J_{SR_{leak_{i}}}
\end{equation}
%
The membrane potential of the \gls{SMC} (in \mV):
\begin{equation} \label{eq:vi}
\begin{split}
\dfrac{\mathrm{d}v_{i}}{\mathrm{d}t} = \gamma_{i}( -J_{Na/K_{i}} - J_{Cl_{i}} - 2J_{VOCC_{i}}- J_{Na/Ca_{i}} - J_{K_{i}} \dots \\
- J_{stretch_{i}} - J_{KIR_{i}} ) +V^{SMC-EC}_{coupling_{i}}
\end{split}
\end{equation}
%
The open state probability of calcium-activated potassium channels (dimensionless):
\begin{equation} \label{eq:dwidt}
\dfrac{\mathrm{d}w_{i}}{\mathrm{d}t} =  \lambda_{i} \left( K_{act_{i}} - w_{i} \right)
\end{equation}
%
The \gls{IP3} concentration om the \gls{SMC} (in \uM):
\begin{equation} \label{eq:dIidt}
\dfrac{\mathrm{d}\IP _{i}}{\mathrm{d}t} = J^{SMC-EC}_{IP_{3}-coupling_{i}} - J_{degrad_{i}}
\end{equation}
%
The \gls{K} concentration in the \gls{SMC} (in \uM):
\begin{equation} \label{eq:dkidt}
\dfrac{\mathrm{d} [K^+_{i}]}{\mathrm{d}t}  = J_{Na/K_{i}}  - J_{KIR_{i}} - J_{K_{i}}
\end{equation}

\begin{table}[h!]
\centering
\begin{tabular}{| p{0.09\linewidth} | >{\footnotesize} p{0.6\linewidth} | >{\footnotesize} p{0.17\linewidth} | >{\footnotesize} p{0.02\linewidth} |}
\arrayrulecolor{lightgrey}\hline
$\gamma_{i}$				& The change in membrane potential by a scaling factor					& 1970 \mVpuM	& \cite{Koenigsberger2006} \\
$\lambda_{i} $				& The rate constant for opening											& 45.0 \pers 	& \cite{Koenigsberger2006} \\
%$\CaConsc$      		& The cytololic [Ca$^{2+}$] in the SMC    								& var. \uM		& - \\
\hline
\end{tabular}
\label{tab:dcidt}
\end{table}

\paragraph{Endothelial Cell}~\\
%
The cytosolic \gls{Ca} concentration in the \gls{EC} (in \uM):
\begin{equation} \label{eq:cj}
\begin{split}
\dfrac{\mathrm{d}\CaConec}{\mathrm{d}t} = J_{IP_{3j}} - J_{ER_{uptake_{j}}} + J_{CICR_{j}} - J_{extrusion_{j}}\dots \\
 + J_{ER_{leak_{j}}} + J_{cation_{j}} + J_{0_{j}} + J_{stretch_{j}} - J_{Ca^{2+}-coupling_{j}}^{SMC-EC}
\end{split}
\end{equation}
%
The \gls{Ca} concentration in the \gls{ER} in the \gls{EC} (in \uM): %copied from SMC
\begin{equation} \label{eq:sj}
\dfrac{\mathrm{d}\CaConee}{\mathrm{d}t} =  J_{SR_{uptake_{j}}} - J_{CICR_{j}} - J_{SR_{leak_{j}}}
\end{equation}
%
The membrane potential of the \gls{EC} (in \mV):
\begin{equation} \label{eq:dvjdt}
\dfrac{\mathrm{d}v_{j}}{\mathrm{d}t} =-\frac{1}{C_{m_{j}}} ( J_{K_{j}}+J_{R_{j}}) + V^{SMC-EC}_{coupling_{j}}
\end{equation}
%
The \gls{IP3} concentration of the \gls{EC} (in \uM):
\begin{equation} \label{eq:dIjdt}
\dfrac{\mathrm{d}\IP_{j}}{\mathrm{d}t} =  J_{agonist_{j}}- J_{degrad_{j}}  - J^{SMC-EC}_{IP_{3}-coupling_{j}}
\end{equation}

\begin{table}[h!]
\centering
\begin{tabular}{| p{0.09\linewidth} | >{\footnotesize} p{0.6\linewidth} | >{\footnotesize} p{0.17\linewidth} | >{\footnotesize} p{0.02\linewidth} |}
\arrayrulecolor{lightgrey}\hline
 $C_{m_{j}}$				& Membrane capacitance												& 25.8  \pF		& \cite{Koenigsberger2006} \\
 
\hline
\end{tabular}
\label{tab:CIP3j}
\end{table}
%\\

\subsubsection{Fluxes}
%
\paragraph{Smooth Muscle Cell}~\\
%
The release of calcium from IP$_{3}$ sensitive stores in the SMC (in \uMps):
\begin{equation} \label{eq:IP3i}
J_{IP_{3i}} = F_{i}\frac{\IP_{i}^{2}}{K_{ri}^{2}+\IP_{i}^{2}}
\end{equation}
%
\begin{table}[h!]
\centering
\begin{tabular}{| p{0.09\linewidth} | >{\footnotesize} p{0.6\linewidth} | >{\footnotesize} p{0.17\linewidth} | >{\footnotesize} p{0.02\linewidth} |}
\arrayrulecolor{lightgrey}\hline
 $F_{i}$      			& Maximal rate of activation-dependent calcium influx			& 0.23 \uMps				& \cite{Koenigsberger2006} \\
$K_{ri}$				& Half-saturation constant for agonist-dependent calcium entry	& 1 \uM					& \cite{Koenigsberger2006} \\
\hline
\end{tabular}
\label{tab:IP3i}
\end{table}
\\
%
\newpage The uptake of calcium into the sarcoplasmic reticulum (in \uMs):
\begin{equation} \label{eq:JSRuptakei}
J_{SR_{uptake_{i}}} = B_{i}\frac{\CaConsc^{2}}{c_{bi}^{2}+\CaConsc^{2}}
\end{equation}
%
\begin{table}[h!]
\centering
\begin{tabular}{| p{0.09\linewidth} | >{\footnotesize} p{0.6\linewidth} | >{\footnotesize} p{0.17\linewidth} | >{\footnotesize} p{0.02\linewidth} |}
\arrayrulecolor{lightgrey}\hline
$B_{i}$      			& SR uptake rate constant							& 2.025 \uMs				& \cite{Koenigsberger2006} \\
$c_{bi}$				& Half-point of the SR ATPase activation sigmoidal (oder Michaelis (Menten) constant) 	& 1.0 \uM					& \cite{Koenigsberger2006} \\
\hline
\end{tabular}
\label{tab:JSRuptakei}
\end{table}
\\
%
The calcium-induced calcium release (CICR; in \uMs):
\begin{equation} \label{eq:JCICRi}
J_{CICR_{i}} = C_{i}\frac{\CaConse^{2}}{s_{ci}^{2}+\CaConse^{2}}    \frac{\CaConsc^{4}}{c_{ci}^{4}+\CaConsc^{4}}
\end{equation}
%
\begin{table}[h!]
\centering
\begin{tabular}{| p{0.09\linewidth} | >{\footnotesize} p{0.6\linewidth} | >{\footnotesize} p{0.17\linewidth} | >{\footnotesize} p{0.02\linewidth} |}
\arrayrulecolor{lightgrey}\hline
$C_{i}$      			& CICR rate constant									& 55 \uMs		& \cite{Koenigsberger2006} \\
$s_{ci}$				& Half-point of the CICR Ca$^{2+}$ efflux sigmoidal			& 2.0 \uM		& \cite{Koenigsberger2006} \\
$c_{ci}$				& Half-point of the CICR activation sigmoidal			& 0.9 \uM		& \cite{Koenigsberger2006} \\
\hline
\end{tabular}
\label{tab:JCICRi}
\end{table}
\\
%
The calcium extrusion by Ca$^{2+}$-ATPase pumps (in \uMs):
\begin{equation} \label{eq:Jextrusioni}
J_{extrusion_{i}} = D_{i}\CaConsc   \left( 1+ \frac{v_{i}-v_{d}}{R_{di}}\right)
\end{equation}
%
\begin{table}[h!]
\centering
\begin{tabular}{| p{0.09\linewidth} | >{\footnotesize} p{0.6\linewidth} | >{\footnotesize} p{0.17\linewidth} | >{\footnotesize} p{0.02\linewidth} |}
\arrayrulecolor{lightgrey}\hline
$D_{i}$      			& Rate constant for Ca$^{2+}$ extrusion by the ATPase pump		 & 0.24	\pers			& \cite{Koenigsberger2005} \\
$v_{d}$					& Intercept of voltage dependence of extrusion ATPase			 & -100.0 \mV			& \cite{Koenigsberger2006} \\
$R_{di}$				& Slope of voltage dependence of extrusion ATPase.				 & 250.0 \mV			& \cite{Koenigsberger2006} \\
\hline
\end{tabular}
\label{tab:Jextrusioni}
\end{table}
\\
%
The leak current from the SR (in \uMs):
\begin{equation} \label{eq:JSRleaki}
J_{SR_{leak_{i}}} = L_{i}\CaConse
\end{equation}
\begin{table}[h!]
\centering
\begin{tabular}{| p{0.09\linewidth} | >{\footnotesize} p{0.6\linewidth} | >{\footnotesize} p{0.17\linewidth} | >{\footnotesize} p{0.02\linewidth} |}
\arrayrulecolor{lightgrey}\hline
$L_{i}$      			& Leak from SR rate constant						 & 0.025 \pers				& \cite{Koenigsberger2006} \\
\hline
\end{tabular}
\label{tab:Jleaki}
\end{table}
\\
%
\newpage
The calcium influx through VOCCs (in \uMs): 
\begin{equation} \label{eq:JVOCCi}
J_{VOCC_{i}} = G_{Cai} \frac{v_{i}-v_{Ca_{1i}}}     {1+ exp(-\left[ \left(  v_{i}-v_{Ca_{2i}}\right) /R_{Cai}      \right] )}
\end{equation}
\begin{table}[h!]
\centering
\begin{tabular}{| p{0.09\linewidth} | >{\footnotesize} p{0.57\linewidth} | >{\footnotesize} p{0.2\linewidth} | >{\footnotesize} p{0.02\linewidth} |}
\arrayrulecolor{lightgrey}\hline
$G_{Cai}$      	& Whole-cell conductance for VOCCs	 					& 1.29e-3  \uMpmVs					& \cite{Koenigsberger2006} \\
$v_{Ca_{1i}}$   & Reversal potential for VOCCs	 						& 100.0 \mV							& \cite{Koenigsberger2006} \\
$v_{Ca_{2i}}$  	& Half-point of the VOCC activation sigmoidal		 	& -24.0 \mV							& ME \\
$R_{Cai}$      	& Maximum slope of the VOCC	activation sigmoidal		& 8.5 \mV							& \cite{Koenigsberger2006} \\
\hline
\end{tabular}
\label{tab:JVOCCi}
\end{table}
\\
%
The flux of calcium exchanging with sodium in the Na$^{+}$Ca$^{2+}$ exchange (in \uMs): 
\begin{equation} \label{eq:JNaCai}
J_{Na/Ca_{i}} = G_{Na/Ca_{i}} \frac{\CaConsc}     {\CaConsc + c_{Na/Cai}} \left( v_{i}-v_{Na/Ca_{i}} \right)
\end{equation}
%
\begin{table}[h!]
\centering
\begin{tabular}{| p{0.09\linewidth} | >{\footnotesize} p{0.57\linewidth} | >{\footnotesize} p{0.2\linewidth} | >{\footnotesize} p{0.02\linewidth} |}
\arrayrulecolor{lightgrey}\hline
$G_{Na/Cai}$   	& Whole-cell conductance for Na$^{+}$/Ca$^{2+}$ exchange			 		 & 3.16e-3 \uMpmVs	& \cite{Koenigsberger2005} \\
$c_{Na/Cai}$   	& Half-point for activation of Na$^{+}$/Ca$^{2+}$ exchange by Ca$^{2+}$		 & 0.5 \uM			& \cite{Koenigsberger2006} \\
$v_{Na/Cai}$   	& Reversal potential for the Na$^{+}$/Ca$^{2+}$ exchanger					 & -30.0 \mV		& \cite{Koenigsberger2006} \\
\hline
\end{tabular}
\label{tab:JNaCai}
\end{table}
\\
%
The calcium flux through the stretch-activated channels in the SMC (in \uMs): 
\begin{equation} \label{eq:Jstretchi}
\begin{split}
J_{stretch_{i}}= \frac{G_{stretch}}{1+ exp\left(-\alpha_{stretch}  \left(  \frac{\Delta pR}{h} -\sigma_{0}   \right) \right)}  \left(  v_{i}-E_{SAC}   \right) 
\end{split}
\end{equation}
%
\begin{table}[h!]
\centering
\begin{tabular}{| p{0.09\linewidth} | >{\footnotesize} p{0.57\linewidth} | >{\footnotesize} p{0.2\linewidth} | >{\footnotesize} p{0.02\linewidth} |}
\arrayrulecolor{lightgrey}\hline
$G_{stretch}$      		& The whole cell conductance for SACs						& 6.1e-3 \uMpmVs	&\cite{Koenigsberger2006} \\
$\alpha_{stretch}$      & Slope of stress dependence of the SAC activation sigmoidal	& 7.4e-3 \pmmHg	&\cite{Koenigsberger2006} \\
$ \Delta p $			& Pressure difference										& 30 \mmHg			& ME \\
$\sigma_{0}$      		& Half-point of the SAC activation sigmoidal				& 500 \mmHg			&\cite{Koenigsberger2006} \\
$E_{SAC}$      			& The reversal potential for SACs							& -18 \mV			&\cite{Koenigsberger2006} \\
\hline
\end{tabular}
\label{tab:Jstretchi}
\end{table}
\\
%
\newpage
Flux through the sodium potassium pump (in \uMs): 
\begin{equation} \label{eq:J_NaK_i}
J_{NaK_{i}}= F_{NaK}
\end{equation}
%
\begin{table}[h!]
\centering
\begin{tabular}{| p{0.09\linewidth} | >{\footnotesize} p{0.57\linewidth} | >{\footnotesize} p{0.2\linewidth} | >{\footnotesize} p{0.02\linewidth} |}
\arrayrulecolor{lightgrey}\hline
$F_{NaK}$      			& Rate of the potassium influx by the sodium potassium pump 		& 4.32e-2 \uMps 	&\cite{Koenigsberger2006} \\
\hline
\end{tabular}
\label{tab:JNaKi}
\end{table}
\\
Chloride flux through the chloride channel (in \uMs):
\begin{equation} \label{eq:JCli}
J_{Cl_{i}} = G_{Cli} \left(  v_{i} - v_{Cli}  \right) 
\end{equation}
%
\begin{table}[h!]
\centering
\begin{tabular}{| p{0.09\linewidth} | >{\footnotesize} p{0.57\linewidth} | >{\footnotesize} p{0.2\linewidth} | >{\footnotesize} p{0.02\linewidth} |}
\arrayrulecolor{lightgrey}\hline
$G_{Cli}$      			& Whole-cell conductance for Cl$^{-}$ current		& 1.34e-3 \uMpmVs	&\cite{Koenigsberger2006} \\
$v_{Cli}$      			& Reversal potential for Cl$^{-}$ channels.			& -25.0 \mV			&\cite{Koenigsberger2006} \\
\hline
\end{tabular}
\label{tab:JCli}
\end{table}
\\
%
Potassium flux through potassium channel (in \uMs):
\begin{equation} \label{eq:JKi}
J_{K_{i}}= G_{Ki} w_{i} \left(  v_{i} - vK_i  \right) 
\end{equation}
%
\begin{table}[h!]
\centering
\begin{tabular}{| p{0.09\linewidth} | >{\footnotesize} p{0.57\linewidth} | >{\footnotesize} p{0.2\linewidth} | >{\footnotesize} p{0.02\linewidth} |}
\arrayrulecolor{lightgrey}\hline
$G_{Ki}$      			& Whole-cell conductance for K$^{+}$ efflux.			& 4.46e-3 \uMpmVs	&\cite{Koenigsberger2005} \\
$vK_i$      			& Nernst potential										& -94 \mV	&\cite{Koenigsberger2005} \\
\hline
\end{tabular}
\label{tab:JKi}
\end{table}
\\
The flux through KIR channels in the SMC (in \uMs): 
\begin{equation} \label{eq:JKIRi}
J_{KIR_{i}} =  \frac{F_{KIR_{i}} g_{KIR_{i}}}{\gamma_{i}}( v_{i} - v_{KIR_{i}})
\end{equation}
%
\begin{table}[h!]
\centering
\begin{tabular}{| p{0.09\linewidth} | >{\footnotesize} p{0.57\linewidth} | >{\footnotesize} p{0.2\linewidth} | >{\footnotesize} p{0.02\linewidth} |}
\arrayrulecolor{lightgrey}\hline
$ F_{KIR_{i}} $ & Scaling factor of potassium efflux through the KIR channel & 7.5e2 & \cite{LoesEvert} \\
\hline
\end{tabular}
\label{tab:JKIRi}
\end{table}
\\
The IP$_{3}$ degradation (in \uMs): 
\begin{equation} \label{eq:Jdegradi}
J_{degrad_{i}}= k_{di}I_{i}
\end{equation}
\begin{table}[h!]
\centering
\begin{tabular}{| p{0.09\linewidth} | >{\footnotesize} p{0.57\linewidth} | >{\footnotesize} p{0.2\linewidth} | >{\footnotesize} p{0.02\linewidth} |}
\arrayrulecolor{lightgrey}\hline
$k_{di}$      			& Rate constant of IP$_{3}$ degradation	& 0.1 \pers	&\cite{Koenigsberger2006} \\
\hline
\end{tabular}
\label{tab:Jdegradi}
\end{table}
%
\newpage
\paragraph{Endothelial Cell}~\\
\\
%
The release of calcium from IP$_{3}$-sensitive stores in the EC (in \uMps):
\begin{equation} \label{eq:JIP3j}
J_{IP_{3j}} = F_{j}\frac{\IP_{j}^{2}}{K_{rj}^{2}+\IP_{j}^{2}}
\end{equation}
\begin{table}[h!]
\centering
\begin{tabular}{| p{0.09\linewidth} | >{\footnotesize} p{0.6\linewidth} | >{\footnotesize} p{0.17\linewidth} | >{\footnotesize} p{0.02\linewidth} |}
\arrayrulecolor{lightgrey}\hline
 $F_{j}$      			& Maximal rate of activation-dependent calcium influx			& 0.23 \uMps				& \cite{Koenigsberger2006} \\
$K_{rj}$				& Half-saturation constant for agonist-dependent calcium entry	& 1 \uM					& \cite{Koenigsberger2006} \\
\hline
\end{tabular}
\label{tab:IP3j}
\end{table}
\\
%
The uptake of calcium into the endoplasmic reticulum (in \uMs):
\begin{equation} \label{eq:JERuptakej}
J_{ER_{uptake_{j}}} = B_{j}\frac{\CaConec^{2}}{c_{bj}^{2}+\CaConec^{2}}
\end{equation}
%
\begin{table}[h!]
\centering
\begin{tabular}{| p{0.09\linewidth} | >{\footnotesize} p{0.6\linewidth} | >{\footnotesize} p{0.17\linewidth} | >{\footnotesize} p{0.02\linewidth} |}
\arrayrulecolor{lightgrey}\hline
$B_{j}$      			& ER uptake rate constant							& 0.5 \uMs				& \cite{Koenigsberger2006} \\
$c_{bj}$				& Half-point of the SR ATPase activation sigmoidal (oder Michaelis (Menten) constant) 	& 1.0 \uM					& \cite{Koenigsberger2006} \\
\hline
\end{tabular}
\label{tab:JERuptakej}
\end{table}
\\
%
The calcium-induced calcium release (CICR; in \uMs):
\begin{equation} \label{eq:JCICRJ}
J_{CICR_{j}} = C_{j}\frac{\CaConee^{2}}{s_{cj}^{2}+\CaConee^{2}}    \frac{\CaConec^{4}}{c_{cj}^{4}+\CaConec^{4}}
\end{equation}
%
\begin{table}[h!]
\centering
\begin{tabular}{| p{0.09\linewidth} | >{\footnotesize} p{0.6\linewidth} | >{\footnotesize} p{0.17\linewidth} | >{\footnotesize} p{0.02\linewidth} |}
\arrayrulecolor{lightgrey}\hline
$C_{j}$      			& CICR rate constant									& 5 \uMs		& \cite{Koenigsberger2006} \\
$s_{cj}$				& Half-point of the CICR Ca$^{2+}$ efflux sigmoidal			& 2.0 \uM		& \cite{Koenigsberger2006} \\
$c_{cj}$				& Half-point of the CICR activation sigmoidal			& 0.9 \uM		& \cite{Koenigsberger2006} \\
\hline
\end{tabular}
\label{tab:JCICRj}
\end{table}
\\
The calcium extrusion by Ca$^{2+}$-ATPase pumps (in \uMs):
\begin{equation} \label{eq:Jextrusionj}
J_{extrusion_{j}} = D_{j}\CaConec 
\end{equation}
%
%
\begin{table}[h!]
\centering
\begin{tabular}{| p{0.09\linewidth} | >{\footnotesize} p{0.6\linewidth} | >{\footnotesize} p{0.17\linewidth} | >{\footnotesize} p{0.02\linewidth} |}
\arrayrulecolor{lightgrey}\hline
$D_{j}$      			& Rate constant for Ca$^{2+}$ extrusion by the ATPase pump		 & 0.24	\pers			& \cite{Koenigsberger2005} \\
\hline
\end{tabular}
\label{tab:Jextrusionj}
\end{table}
\\
%
\newpage 
The calcium flux through the stretch-activated channels in the EC (in \uMs): 
\begin{equation} \label{eq:Jstretchj}
J_{stretch_{j}}= \frac{G_{stretch}}{1+ e^{-\alpha_{stretch}  \left(  \sigma -\sigma_{0}   \right) }}  \left(  v_{j}-E_{SAC}   \right) \\
= \frac{G_{stretch}}{1+ e^{-\alpha_{stretch}  \left(  \frac{\Delta pR}{h} -\sigma_{0}   \right) }}  \left(  v_{j}-E_{SAC}   \right) 
\end{equation}
%
\begin{table}[h!]
\centering
\begin{tabular}{| p{0.09\linewidth} | >{\footnotesize} p{0.6\linewidth} | >{\footnotesize} p{0.17\linewidth} | >{\footnotesize} p{0.02\linewidth} |}
\arrayrulecolor{lightgrey}\hline
$G_{stretch}$      		& The whole cell conductance for SACs						& 6.1e-3 \uMpmVs	&\cite{Koenigsberger2006} \\
$\alpha_{stretch}$      & Slope of stress dependence of the SAC activation sigmoidal	& 7.4e-3 \pmmHg	&\cite{Koenigsberger2006} \\
$ \Delta p $			& Pressure difference										& 30 \mmHg			& ME \\
$\sigma_{0}$      		& Half-point of the SAC activation sigmoidal				& 500 \mmHg			&\cite{Koenigsberger2006} \\
$E_{SAC}$      			& The reversal potential for SACs							& -18 \mV			&\cite{Koenigsberger2006} \\
\hline
\end{tabular}
\label{tab:Jstretchj}
\end{table}
\\
%
The leak current from the ER (in \uMs):
\begin{equation} \label{eq:JERleakj}
J_{ER_{leak_{j}}} = L_{j}\CaConee
\end{equation}
%
\begin{table}[h!]
\centering
\begin{tabular}{| p{0.09\linewidth} | >{\footnotesize} p{0.6\linewidth} | >{\footnotesize} p{0.17\linewidth} | >{\footnotesize} p{0.02\linewidth} |}
\arrayrulecolor{lightgrey}\hline
$L_{j}$      			& Rate constant for Ca$^{2+}$ leak from the ER 		 & 0.025	\pers			& \cite{Koenigsberger2006} \\
\hline
\end{tabular}
\label{tab:JERleakj}
\end{table}
\\
%
The calcium influx through nonselective cation channels (in \uMs):
\begin{equation} \label{eq:Jcationj}
J_{cation_{j}} = G_{cat_{j}} (E_{Ca_{j}} - v_{j}) \frac{1}{2} \left(   1+ \mathrm{tanh}  \left(  \frac{\mathrm{log}_{10} \CaConec - m_{3_{cat_{j}}} }    {m_{4_{cat_{j}}}}   \right)      \right) 
\end{equation}
%
%
\begin{table}[h!]
\centering
\begin{tabular}{| p{0.09\linewidth} | >{\footnotesize} p{0.6\linewidth} | >{\footnotesize} p{0.17\linewidth} | >{\footnotesize} p{0.02\linewidth} |}
\arrayrulecolor{lightgrey}\hline
$G_{cat j}$      		& Whole-cell cation channel conductivity						 	& 6.6e-4 \uMpmVs	& \cite{Koenigsberger2006} \\
$E_{Caj}$      			& Ca$^{2+}$ equilibrium potential								 	& 50 \mV		& \cite{Koenigsberger2006} \\

$m_{3_{catj}}$      	& Model constant				 	& -6.18 \uM		& \cite{Koenigsberger2006} \\
$m_{4_{catj}}$      	& Model constant					& 0.37  \uM		& \cite{Koenigsberger2006} \\
\hline
\end{tabular}
\label{tab:Jcationj}
\end{table}
\\
%
The potassium efflux through the $J_{BK_{Caj}}$ channel and the $J_{SK_{Caj}}$ channel (in \uMs):
\begin{equation} \label{eq:JKj}
J_{K_{j}} = G_{totj} (v_{j}-v_{Kj}) \left(   J_{BK_{Caj}} + J_{SK_{Caj}} \right) 
\end{equation}
%
%
\begin{table}[h!]
\centering
\begin{tabular}{| p{0.09\linewidth} | >{\footnotesize} p{0.6\linewidth} | >{\footnotesize} p{0.17\linewidth} | >{\footnotesize} p{0.02\linewidth} |}
\arrayrulecolor{lightgrey}\hline
$G_{totj}$      		& Total potassium channel conductivity.						 		& 6927 \pS		& \cite{Koenigsberger2006} \\
$v_{Kj}$      			& K$^{+}$ equilibrium potential					 			 		& -80.0 \mV		& \cite{Koenigsberger2006} \\
\hline
\end{tabular}
\label{tab:JKj}
\end{table}
\\
%
The potassium efflux through the $J_{BK_{Caj}}$ channel (in \uMs):
\begin{equation} \label{eq:JBKCAj}
J_{BK_{Caj}} = 0.2 \left(   1+ \mathrm{tanh}   \left(   \frac{   (\mathrm{log}_{10} \CaConec - c) (v_{j}-b_{j}) - a_{1j}  }   { m_{3bj} ( v_{j} + a_{2j} (\mathrm{log}_{10} \CaConec -c )-b_{j} )^{2} + m_{4bj} }  \right)     \right)  
\end{equation}
%
The potassium efflux through the $J_{SK_{Caj}}$ channel (in \uMs):
\begin{equation} \label{eq:JSKCaj}
J_{SK_{Caj}} = 0.3\left( 1+ \mathrm{tanh}  \left(  \frac{   \mathrm{log}_{10} \CaConec -m_{3sj}  } {m_{4sj}}  \right)      \right) 
\end{equation}
%
\begin{table}[h!]
\centering
\begin{tabular}{| p{0.09\linewidth} | >{\footnotesize} p{0.6\linewidth} | >{\footnotesize} p{0.17\linewidth} | >{\footnotesize} p{0.02\linewidth} |}
\arrayrulecolor{lightgrey}\hline
$c$      				& Model constant, further explanation see reference					& -6.4 \uM			& \cite{Koenigsberger2006} \\
$b_{j}$      			& Model constant, further explanation see reference					& -80.8 \mV		& \cite{Koenigsberger2006} \\
$a_{1j}$      			& Model constant, further explanation see reference					& 53.3 \uMkeermV	& \cite{Koenigsberger2006} \\
$a_{2j}$      			& Model constant, further explanation see reference					& 53.3 \mVpuM		& \cite{Koenigsberger2006} \\
$m_{3bj}$      			& Model constant, further explanation see reference					& 1.32e-3 \uMpmV	& \cite{Koenigsberger2006} \\
$m_{4bj}$      			& Model constant, further explanation see reference					& 0.30	\uMkeermV	& \cite{Koenigsberger2006} \\
$m_{3sj}$      			& Model constant, further explanation see reference					& -0.28 \uM		& \cite{Koenigsberger2006} \\
$m_{4sj}$      			& Model constant, further explanation see reference					& 0.389 \uM		& \cite{Koenigsberger2006} \\
\hline
\end{tabular}
\label{tab:JBKCAj}
\end{table}
\\
%
The residual current regrouping chloride and sodium current flux (in \uMs):
\begin{equation} \label{eq:JRj}
J_{R_{j}} = G_{R_{j}} ( v_{j} - v_{rest j}  )
\end{equation}
%
\begin{table}[h!]
\centering
\begin{tabular}{| p{0.09\linewidth} | >{\footnotesize} p{0.6\linewidth} | >{\footnotesize} p{0.17\linewidth} | >{\footnotesize} p{0.02\linewidth} |}
\arrayrulecolor{lightgrey}\hline
$G_{R_{j}}$      		& Residual current conductivity										& 955 \pS			& \cite{Koenigsberger2006} \\
$v_{rest j}$      		& Membrane resting potential						 				& -31.1 \mV		& \cite{Koenigsberger2006} \\
\hline
\end{tabular}
\label{tab:JRj}
\end{table}
\\
%
The IP$_{3}$ degradation (in \uMs):  
\begin{equation} \label{eq:Jdegradj}
J_{degrad_{j}}= k_{dj} \IP_{j}
\end{equation}
%
\begin{table}[h!]
\centering
\begin{tabular}{| p{0.09\linewidth} | >{\footnotesize} p{0.6\linewidth} | >{\footnotesize} p{0.17\linewidth} | >{\footnotesize} p{0.02\linewidth} |}
\arrayrulecolor{lightgrey}\hline
$k_{dj}$      			& Rate constant of IP$_{3}$ degradation						 		& 0.1 \pers		& \cite{Koenigsberger2006} \\
\hline
\end{tabular}
\label{tab:Jdegradj}
\end{table}
\\
%
%
\subsubsection{Coupling}~\\
%
The heterocellular electrical coupling between SMCs en ECs (in \mVs):
\begin{equation} \label{eq:Vcouplingi}
V_{coupling_{i}}^{SMC-EC}= -G_{coup}(v_{i}-v_{j})
\end{equation}
%
The heterocellular IP$_{3}$ coupling between SMCs and ECs (in \uMs):
\begin{equation} \label{eq:JIP3couplingi}
J_{IP_{3}-coupling_{i}}^{SMC-EC}= -P_{IP_{3}}(\IP_{i}-\IP_{j})
\end{equation}
%
Calcium coupling with EC (in\uMs):
\begin{equation} \label{eq:JCAcouplingi}
J_{Ca^{2+}-coupling_{i}}^{SMC-EC}= -P_{Ca^{2+}}(\CaConsc-\CaConec)
\end{equation}
%
\begin{table}[h!]
\centering
\begin{tabular}{| p{0.09\linewidth} | >{\footnotesize} p{0.6\linewidth} | >{\footnotesize} p{0.17\linewidth} | >{\footnotesize} p{0.02\linewidth} |}
\arrayrulecolor{lightgrey}\hline
$P_{CA^{2+}}$      		& Heterocellular $P_{Ca^{2+}}$ coupling coefficient	& 0.05 \pers	&  \cite{Koenigsberger2006} \\
$P_{IP_{3}}$      		& Heterocellular IP$_{3}$ coupling coefficient	& 0.05 \pers	&  \cite{Koenigsberger2006} \\
$G_{coup}$      		& Heterocellular electrical coupling coefficient		& 5 \pers	& ME \\
\hline
\end{tabular}
\label{tab:JCA3couplingi}
\end{table}
%
\newpage
\subsubsection{Additional Equations}
%
The equilibrium distribution of open channel states for the voltage and calcium activated potassium channels (dimensionless):
\begin{equation} \label{eq:Kacti}
K_{act_{i}}= \frac{  \left( \CaConsc + c_{wi}\right)^{2}}    {\left( \CaConsc + c_{wi} \right)^{2}    + \beta_{i} exp( -\left(   \left[ v_{i}-v_{Ca_{3i}}\right] /R_{Ki}   \right) )      }
\end{equation}
%
Nernst potential of the KIR channel in the SMC (in mV):
\begin{equation}\label{eq:vKIR}
v_{KIR_i} = z_1 K_p-z_2
\end{equation}
%
Conductance of KIR channel (in  s$^{-1} $):
\begin{equation}\label{eq:gKIR}
g_{KIR_i} = exp(z_5v_i +z_3 K_p - z_4)
\end{equation}
%
%
%
\begin{table}[h!]
\centering
\begin{tabular}{| p{0.09\linewidth} | >{\footnotesize} p{0.6\linewidth} | >{\footnotesize} p{0.17\linewidth} | >{\footnotesize} p{0.02\linewidth} |}
\arrayrulecolor{lightgrey}\hline
$c_{wi}$      			& Translation factor for Ca$^{2+}$ dependence of K$_{Ca}$ channel activation sigmoidal.	& 0.0  \uM	&\cite{Koenigsberger2006} \\
$\beta_{i}$     		& Translation factor for membrane potential dependence of K$_{Ca}$ channel activation sigmoidal.	& 0.13 \uMtwee& \cite{Koenigsberger2006} \\
$v_{Ca_{3i}}$   		& Half-point for the K$_{Ca}$ channel activation sigmoidal.			& -27 \mV	&\cite{Koenigsberger2006} \\
$R_{Ki}$      			& Maximum slope of the K$_{Ca}$ activation sigmoidal.				& 12 \mV	&\cite{Koenigsberger2006} \\
$z_{1}$      			& Model estimation for membrane voltage KIR channel				& 4.5e3 \mV	&\cite{LoesEvert}  \\
$z_{2}$      			& Model estimation for membrane voltage KIR channel			& 112 \mV	&\cite{LoesEvert}  \\
$z_{3}$      			& Model estimation for the KIR channel conductance				& 4.2e2 \uMpmVs	&\cite{LoesEvert}  \\
$z_{4}$      			& Model estimation for the KIR channel conductance				& 12.6 \uMpmVs	&\cite{LoesEvert}  \\
$z_{5}$      			& Model estimation for the KIR channel conductance			& -7.4e-2 \uMpmVs	&\cite{LoesEvert}  \\
\hline
\end{tabular}
\label{tab:Addeq}
\end{table}
\\
%
\newpage
\subsection{Contraction and Mechanical Model}

\subsubsection{Contraction Equations}
%
The fraction of free phosphorylated cross-bridges (dimensionless):
\begin{equation} \label{eq:dMpdt}
\frac{\dd[Mp]}{\dd t} = K_{4}[AMp] +K_{1} [M] - ( K_{2} + K_{3} ) [Mp]
\end{equation}
%
The fraction of attached phosphorylated cross-bridges (dimensionless):
\begin{equation} \label{eq:dAMpdt}
\frac{\dd[AMp]}{\dd t} =K_{3} [Mp] + K_{6} [AM] - ( K_{4} + K_{5} )[AMp]
\end{equation} 
%
The fraction of attached dephosphorylated cross-bridges (dimensionless):
\begin{equation} \label{eq:dAMdt}
\frac{\dd[AM]}{\dd t} = K_{5} [AMp]-(K_{7}+K_{6})[AM]
\end{equation}
%
The fraction of free non-phosphorylated cross-bridges (dimensionless):
\begin{equation} \label{eq:dMdt}
[M]=1-[AM]-[AMp]-[Mp]
%\frac{\dd[M]}{\dd t} = -K_{1_{i}} [M] + K_{2_{i}} [Mp] + K_{7_{i}} [AM]
\end{equation}
%
The rate constants that represent phosphorylation of M to Mp and of AM to AMp by the active myosin light chain kinase (MLCK), respectively (in \pers):
\begin{equation} \label{eq:gamma}
K_{1} = K_{6} = \gamma_{cross} \CaConsc ^{3}
\end{equation}
%
%Note that:
%\begin{equation} \label{eq:fractiesone}
%[AM]+[AMp]+[Mp]+[M]=1
%\end{equation}
%
\begin{table}[h!]
\centering
\begin{tabular}{| p{0.09\linewidth} | >{\footnotesize} p{0.6\linewidth} | >{\footnotesize} p{0.17\linewidth} | >{\footnotesize} p{0.02\linewidth} |}
\arrayrulecolor{lightgrey}\hline
$K_{2}$      	& The rate constant for dephosphorylation (of Mp to M) by myosin light-chain phosphatase (MLCP)																			 & 0.5 \pers & \cite{Hai1989} \\
$K_{3}$      	& The rate constants representing the attachment/detachment of fast cycling phosphorylated crossbridges																	 & 0.4 \pers	& \cite{Hai1989} \\
$K_{4}$      	& The rate constants representing the attachment/detachment of fast cycling phosphorylated crossbridges 																	 & 0.1 \pers	& \cite{Hai1989} \\
$K_{5}$      & The rate constant for dephosphorylation (of AMp to AM) by myosin light-chain phosphatase (MLCP)																			 & 0.5 \pers	& \cite{Hai1989} \\
$K_{7}$      	& The rate constant for latch-bridge detachment					& 0.1 \pers	& \cite{Hai1989} \\
$\gamma_{cross}$      	& The sensitivity of the contractile apparatus to calcium		& 17 \puMdries	& \cite{Hai1989} \\
%$n_{cross}$      		& A constant for further explanation see reference				& 3 \Dless	& \cite{Hai1988} \\
\hline
\end{tabular}
%\caption{This table shows some data}
\label{tab:crossbridge}
\end{table}
\newpage
\subsubsection{Mechanical Equations}

The wall thickness of the vessel (in \um):
\begin{equation} \label{eq:h2}
h=-0.1R
% h = -R+\sqrt{R^2+2R_{0_{pas}}h_{0_{pas}}+h_{0_{pas}}^2}
\end{equation}
%
The fraction of attached myosin cross-bridges (dimensionless):
\begin{equation}
F_r = [AM_p] + [AM]
\end{equation}
%
The vessel radius (in \um):
\begin{equation} \label{eq:dRdt2}
\dfrac{\mathrm{d}R}{\mathrm{d}t}= \frac{R_{0_{pas}}}{\eta}\left(   \frac{ R P_{T}}{h}  - E(F_r) \frac{R - R_0(F_r)}{R_0(F_r)} \right)
\end{equation}
%
with:
\begin{equation}
E(F_r)= E_{pas} + F_r \left(E_{act} - E_{pas} \right)
\end{equation}
%
\begin{equation}
R_0(F_r)=R_{0_{pas}} + F_r (\alpha -1) R_{0_{pas}}
\end{equation}
%
\begin{table}[h!]
\centering
\begin{tabular}{| p{0.09\linewidth} | >{\footnotesize} p{0.6\linewidth} | >{\footnotesize} p{0.17\linewidth} | >{\footnotesize} p{0.02\linewidth} |}
\arrayrulecolor{lightgrey}\hline
$\eta   $				& viscosity															& 1e4 Pa s 		&  \cite{Koenigsberger2006}\\
$R_{0_{pas}}$			& Radius of the vessel when passive and no stress is applied		& 20  \um 		& ME \\
$h_{0_{pas}}$			& Wall thickness when passive and no stress is applied				& 3   \um		& ME \\
$P_T$					& Transmural pressure												& 4000 \Pa		& ME \\
${E}_{pas}$				& Young's moduli for the passive vessel								& 66e3 \Pa 		&  \cite{Gore}\\
${E}_{act}$				&  Young's moduli for the active vessel	& 233e3 \Pa 	& \cite{Gore}\\
$\alpha$				& Scaling factor initial radius										& 0.6    		& \cite{Gore}\\
\hline
\end{tabular}
%\caption{This table shows some data}
\label{tab:radius}
\end{table}


