\section{Release notes}

\subsection{Adaptations to the astrocyte model}
In the creation of NVU 1.1 the work of Farr\& David(2011) \cite{Farr2011} was implemented in NVU 1.0. This means that several pathways were added and equations were combined. First a glutamate release in the synaptic cleft was simulated by creating a smooth pulse function, $\rho$, describing the ratio of bound to total glutamate receptors on the synapse end of the astrocyte. This induces an \gls{IP3} release into the cell, causing the release of Calcium from the \gls{ER} into the cytosol, which on his turn leads to the production of EET. The open state of the BK-channels in NVU 1.0 depends only on the membrane voltage, in NVU 1.1 the opening of BK-channels is regulated by the membrane voltage as well as the EET and \gls{Ca} concentration. Furthermore several corrections, proposed by de Kock\& van der Donck(2013) \cite{LoesEvert}, on the model by Farr were included. These corrections include an equation to describe the buffering parameter in the \gls{Ca} conservation equation (Farr \cite{Farr2011} used a constant to describe this parameter) and some changes in parametervalues.

\subsection{New/changed equations \& parameters for the astrocyte model}
\paragraph{$\rho$ input signal}

The smooth pulse function $\rho$
\begin{equation}
\rho(t) = \frac{Amp - base}{2}\times\left(1+\mathrm{tanh}\left(\frac{t-t_0}{\theta_L}\right)\right)+base+\frac{Amp-base}{2}\times\left(1+\mathrm{tanh}\left(\frac{t-t_2}{\theta_R}\right)\right)+base-Amp     
\end{equation}
%
\begin{table}[h!]
	\centering
	\begin{tabular}{| p{0.09\linewidth} | >{\footnotesize} p{0.6\linewidth} | >{\footnotesize} p{0.17\linewidth} | >{\footnotesize} p{0.02\linewidth} |}
		\arrayrulecolor{lightgrey}\hline
		$Amp$           & Amplitude of smooth pulse function & 0.7 & ME\\
		$Base$          & Baseline of smooth pulse function & 0.1 & ME\\
		$\theta_L$      & Left ramp of smooth pulse function & 1 & ME\\
		$\theta_R$      & Left ramp of smooth pulse function & 1 & ME\\
		\hline
	\end{tabular}
\end{table}

\subsubsection{Conservation Equations}
\gls{Ca} concentration in the astrocytic cytosol:
\begin{equation} \label{eq:ckInt}
\dfrac{\mathrm{d}c_k}{\mathrm{d}t}= B_{\mathrm{cyt}}(J_{\mathrm{IP_3}}-J_{\mathrm{pump}}+J_{\mathrm{ER_leak}})
\end{equation}
%
\gls{Ca} concentration in the astrocytic \gls{ER}:
\begin{equation} \label{eq:skInt}
\dfrac{\mathrm{d}s_k}{\mathrm{d}t}= \frac{1}{VR_{\mathrm{ER_cyt}}}(\frac{dc_k}{dt})
\end{equation}
%
The inactivation variable for \gls{IP3}:
\begin{equation} \label{eq:hkInt}
\dfrac{\mathrm{d}h_k}{\mathrm{d}t}= k_{\mathrm{on}}[K_{\mathrm{inh}}-(c_k+K_{\mathrm{inh}})h_k]
\end{equation}
%
The \gls{IP3} concentration:
\begin{equation} \label{eq:ikInt}
\dfrac{\mathrm{d}i_k}{\mathrm{d}t}= r_hG-k_{\mathrm{deg}}i_k
\end{equation}
%
The EET concentration:
\begin{equation} \label{eq:eetkInt}
\dfrac{\mathrm{d}eetk_k}{\mathrm{d}t}= V_{\mathrm{eet}}(c_k-c_{\mathrm{k,min}})-k_{\mathrm{eet}}eet_k
\end{equation}
%
Open probability of the BK channel (\pers):
\begin{equation} \label{eq:dwkdt}
\frac{\mathrm{d}w_{k}}{\mathrm{d}t} = \phi_{w} \left(w_{\infty}-w_{k} \right) 
\end{equation}

\begin{table}[h!]
	\centering
	\begin{tabular}{| p{0.09\linewidth} | >{\footnotesize} p{0.6\linewidth} | >{\footnotesize} p{0.17\linewidth} | >{\footnotesize} p{0.02\linewidth} |}
		\arrayrulecolor{lightgrey}\hline
		$ VR_{\mathrm{ER_{\mathrm{cyt}}}} $  & Volume ratio of the \gls{ER} to the cytosol in the astrocyte  & 0.185 [-] & \cite{Farr2011} \\
		$k_{\mathrm{on}} $         & Rate of \gls{Ca} binding to the \gls{IP3}R & 2 [\uMps] & \cite{Farr2011} \\
		$K_{\mathrm{inh}}$          & Dissociation rate of $k_{\mathrm{on}}$  & [0.1\uM] & \cite{Farr2011} \\
		$r_h$                      & Maximum rate of \gls{IP3} production in the astrocyte & 4.8 [\uM] & \cite{Farr2011} \\
		$k_{\mathrm{deg}}$          & Rate constant for \gls{IP3} degradation & 1.25 [\pers] & \cite{Farr2011} \\
		$V_{\mathrm{eet}} $        & Rate constant for EET production   & 72[\uM] & \cite{Farr2011} \\
		$k_{\mathrm{eet}}$         & Rate constant for EET degradation  & 7.2[\uM] & \cite{Farr2011} \\
		$c_{\mathrm{k,min}}$       & Minimum \gls{Ca} concentration required for EET production & 0.1 [uM] & \cite{Farr2011} \\			
		\hline
	\end{tabular}
\end{table}
		
\subsubsection{Fluxes}
\gls{Ca} flux from the ER to the cytosol in the astrocyte through \gls{IP3} Receptors (\gls{IP3}R) by \gls{IP3}: 
\begin{equation} \label{eq:J_ip3}
J_{\mathrm{IP3}}=J_{\mathrm{max}}[(\frac{i_k}{i_k+K_i})(\frac{c_k}{c_k+K_{\mathrm{act}}})h_k]^3\times [1-\frac{c_k}{s_k}] 
\end{equation}
%
The leakage \gls{Ca}  flux from the \gls{ER} to the cytosol in the astrocyte:
\begin{equation} \label{eq:J_ER_leak}
J_{\mathrm{ER_leak}} = P_L(1-\frac{c_k}{s_k})
\end{equation}	
%
The ATP dependent \gls{Ca}  pump flux from the cytoplasm to the ER in the astrocyte:
\begin{equation} \label{eq:J_pump}
J_{\mathrm{pump}} = V_{\mathrm{max}}\frac{c_k^2}{c_k^2+k_pump^2}
\end{equation}
\begin{table}[h!]
	\centering
	\begin{tabular}{| p{0.09\linewidth} | >{\footnotesize} p{0.6\linewidth} | >{\footnotesize} p{0.17\linewidth} | >{\footnotesize} p{0.02\linewidth} |}
		\arrayrulecolor{lightgrey}\hline	
		$J_{max}$       & Maximum \gls{IP3} rate                                                & 2880 \uMps        & \cite{Farr2011} \\  
		$K_I$           & Dissociation constant for \gls{IP3} binding to \gls{IP3}R             & 0.03 \uM          & \cite{Farr2011} \\ 
		$K_{act}$       & Dissociation constant for \gls{Ca} binding to \gls{IP3}R              & 0.17 \uM          & \cite{Farr2011} \\ 
		$P_L$           & Associated with the steady state \gls{Ca} balance                     & 0.0842 \uM           & \cite{LoesEvert} \\ 
		$V_{max}$       & Maximal pumping rate of the \gls{Ca} pump                             & 20 \uMps          & \cite{Farr2011} \\ 
		$k_{pump}$      & Dissociation constant of the \gls{Ca} pump                            & 0.24 \uM          & \cite{Farr2011} \\ 
		\hline
	\end{tabular}
\end{table}
\subsubsection{Additional equations}
The Calcium buffering parameter in the astrocytic cytosol (-)
\begin{equation} \label{eq:B_cyt}
B_{cyt}=\left(1+BK_{end}+ \frac{K_{ex}B_{ex}}{(K_{ex}+c_k)^2}\right)^{-1} 
\end{equation}
The ratio of active to total G-protein (-)
\begin{equation} \label{eq:G}
G=\frac{\rho+\delta}{K_g+\rho+\delta}
\end{equation}
Equilibrium state BK-channel (-):
\begin{equation} \label{eq:winf}
w_{\infty}=0.5 \left(1+\mathrm{tanh}\left(\frac{v_{k}+(eet_{\mathrm{shift}}eet_k)-v_{3} }{v_{4}} \right)  \right) 
\end{equation}
%
The time constant associated with the opening of BK channels	 (in \pers):
\begin{equation} \label{eq:phin}
\phi_{w}=\psi_{w}\mathrm{cosh}\left( \frac{v_{k}-v_{3}}{2v_{4}}\right) 
\end{equation}
\gls{Ca} dependent shift of the opening of the BK-channels
\begin{equation} \label{eq:v_3}
v_{3}=\frac{v_5}{2}\mathrm{tanh}\left( \frac{c_k-Ca_3}{Ca_4}\right)+v_6 
\end{equation}

\begin{table}[h!]
	\centering
	\begin{tabular}{| p{0.09\linewidth} | >{\footnotesize} p{0.6\linewidth} | >{\footnotesize} p{0.17\linewidth} | >{\footnotesize} p{0.02\linewidth} |}
		\arrayrulecolor{lightgrey}\hline
		
		$BK_{end}$      & Cytosolic endogenous buffer constant                              & 40 [-] & \cite{LoesEvert} \\
		$K_{ex}$        & Cytosolic exogenous buffer dissociation constant                  & 0.26 [\uM] & \cite{LoesEvert} \\
		$B_{ex}$        & Concentration of cytosolic exogenous buffer                       & 11.35 [\uM] & \cite{LoesEvert} \\
		$\delta$        & Ratio of the activities of the bound and unbound receptors        & 1.235e-3 [\uM] & \cite{LoesEvert}\\
		$K_G$           & The G-protein dissociation constant                               & 8.82  [uM] & \cite{LoesEvert}\\
		$v_{4}$			& A measure of the spread of the distribution of the open probability of the BK channel	& 14.5e-3 [\Volt]   &  \cite{Gonzalez1994} \\
		$v_{5}$			& Determines the range of the shift of $n_{\inf}$ as calcium varies    	& 8e-3 [\Volt]  & \cite{LoesEvert}  \\
		$v_{6}$			& The voltage associated with the opening of half the population		& -15e-3 [\Volt]  & \cite{LoesEvert}  \\
		$ \psi_{w}$    	& A characteristic time for the open probability of the BK channel		& 2.664 [\pers] & \cite{Gonzalez1994} \\
		
		\hline
	\end{tabular}
\end{table}
